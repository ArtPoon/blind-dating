\documentclass[9pt,twocolumn,twoside,lineno]{pnas-new}
% Use the lineno option to display guide line numbers if required.
% Note that the use of elements such as single-column equations
% may affect the guide line number alignment. 

\templatetype{pnasresearcharticle} % Choose template 
% {pnasresearcharticle} = Template for a two-column research article
% {pnasmathematics} = Template for a one-column mathematics article
% {pnasinvited} = Template for a PNAS invited submission


\usepackage{comment}
\usepackage[capitalize]{cleveref}

\newcommand{\figcaption}[1]{\refstepcounter{figure} \subsection*{Fig. \thefigure{}} #1}
\newcommand{\tabcaption}[1]{\refstepcounter{table}}


\title{Blind-dating: recovering the integration dates of the latent human immunodeficiency virus reservoir using longitudinal samples}

% Use letters for affiliations, numbers to show equal authorship (if applicable) and to indicate the corresponding author
\author[a]{Bradley R. Jones}
\author[b]{Natalie N. Kinoch} 
\author[a]{Joshua Horacsek}
\author[b]{Philip Mwimanzi}
\author[b]{Bemuluyigza Baraki}
\author[c]{John Huang}
\author[c]{Ronald Truong}
\author[a]{Bruce Ganase}
\author[a]{Marianne Harris}
\author[a]{Robert Hollebakken}
\author[a,d]{P. Richard Harrigan}
\author[c]{R. Brad Jones}
\author[a,b]{Mark A. Brockman}
\author[a,d,1]{Jeffrey B. Joy}
\author[e,1]{Art F. Y. Poon}
\author[b,1,2]{Zabrina L. Brumme}

\affil[a]{BC Centre for Excellence in HIV/AIDS, Vancouver, Canada}
\affil[b]{Faculty of Medicine, Simon Fraser University, Burnaby, Canada}
\affil[c]{Department of Microbiology, Immunology and Tropical Medicine, George Washington University, Washington, D.C., USA}
\affil[d]{Department of Medicine, University of British Columbia, Vancouver, Canada}
\affil[e]{Department of Pathology and Laboratory Medicine, Western University, London, Canada}

% Please give the surname of the lead author for the running footer
\leadauthor{Jones} 

% Please add here a significance statement to explain the relevance of your work
\significancestatement{Significance (120 words).}

% Please include corresponding author, author contribution and author declaration information
\authorcontributions{Please provide details of author contributions here.}
\authordeclaration{Please declare any conflict of interest here.}
\equalauthors{\textsuperscript{1}J.B.J.(Jeffrey B. Joy), A.F.Y.P. (Art F. Y. Poon), and Z.L.B (Zabrina L. Brumme) contributed equally to this work.}
\correspondingauthor{\textsuperscript{2}To whom correspondence should be addressed. E-mail: zbrumme@sfu.ca}

% Keywords are not mandatory, but authors are strongly encouraged to provide them. If provided, please include two to five keywords, separated by the pipe symbol, e.g:
\keywords{Human immunodeficiency virus $|$ Viral reservoirs $|$ Phylogenetics $|$ Linear models} 

\begin{abstract}
Abstract (250 words).
\end{abstract}

\dates{This manuscript was compiled on \today}
\doi{\url{www.pnas.org/cgi/doi/10.1073/pnas.XXXXXXXXXX}}

\begin{document}

% Optional adjustment to line up main text (after abstract) of first page with line numbers, when using both lineno and twocolumn options.
% You should only change this length when you've finalised the article contents.
\verticaladjustment{-2pt}

\maketitle{}
\thispagestyle{firststyle}
\ifthenelse{\boolean{shortarticle}}{\ifthenelse{\boolean{singlecolumn}}{\abscontentformatted}{\abscontent}}{}

% If your first paragraph (i.e. with the \dropcap) contains a list environment (quote, quotation, theorem, definition, enumerate, itemize...), the line after the list may have some extra indentation. If this is the case, add \parshape=0 to the end of the list environment.
\dropcap{I}ntroduction.
((Motivation.))

((Idea.))

\section*{Reconstructing integration dates}
Since HIV evolution adheres to the molecular clock hypothesis \cite{Leitner99,Park16}, 
While under viral latency 
See \cref{fig:latenttree}.

To recover the linear molecular clock, one can use a linear model (LM).
Our LM is given by:
\begin{align}
	D &= a + \mu T,\label{eq:glm}
\end{align}
with response: $T$, the collection date; predictor: $D$, the genetic distance from the root; and parameters: $\mu$, the mutation rate and $a$, a constant genetic distance.
Details of the method are given in the Blind-dating subsection of the Materials and Methods.

\section*{Model validation on simulated data}
foo

\section*{Results on patient data}
foo

\subsection*{Comparison of rooting methods}
The correlation between the estimated dates for our patient using the different rooting methods --- outgroup rooting (OGR) and root-to-tip regression (RTT) --- was 0.83 (pearson)/0.81 (spearman) and the root mean square deviation was 970 days.

\section*{Stress tests on patient data}
foo

\section*{Conclusions}
foo

Most other models of HIV latent reservoirs are dynamical models \cite{Rong09,Pace11}. These models can be complicated, including many components and parameters.

\section*{Figure Legends}
\figcaption{Reconstructing the time that a viral lineage entered a latent state from with-host sequence variation.
	A dashed line illustrates the linear relationship between the divergence of lineages from the ancestral sequence at the root ($y$-axis) and passage of time since the root ($x$-axis).
	Grey lines represent the reconstructed phylogenetic relationships among these lineages.
	The lineages were sampled (open circles) at three points in time.
	%Above is a time calibrated phylogeny with three taxa, the dotted line in is an example of latency. 
	One of these lineages had become latent at an earlier point in time (red hexagon).
	This lineage subsequently underwent negligible molecular evolution until it was sampled as integrated viral DNA (dashed circle).
	If the relationship between sequence divergence and time is sufficiently linear, then the time between latency establishment and sampling, here represented by the thick red dashed line, can be iInferred fomormation for Authors, its sequence.                                                                                                                                    hived at time $t$, and was collected at the same time as sequence B, at time $t^\prime$ (a drift of $t^\prime - t$ is expected).
	%If the molecular clock assumption holds true, the MRCA is known, and B and C are reliable time-points, then the date at $t$ can be inferred from from the expected amount of evolution for B from the root.
}
\label{fig:latenttree}

\matmethods{
\subsection*{Data collection, sequencing and alignment}
((Sequencing.))
The sequences were aligned using MUSCLE v3.8.31 \cite{muscle} and inspected and cleaned with AliView v1.18 \cite{aliview}.
For each pair of identical sequences, we retained only one copy.
There was one instance where there was a triplet of RNA sequences that were collected at two different time points.
For this triplet we kept one sequence from the earlier time point.
\subsection*{Simulated data generation}
To characterize sources of error in reconstructing dates on tips of a phylogeny, we generated phylogenetic trees by direct simulation.
We generated phylogenies with 50 tips each under a birth-death model with serial sampling using the sim.bdsky.stt function in the R package TreeSim v2.3 \cite{treesim}.
We estimated the parameters to use for the birth-death model by fitting the HIV RNA sequences from Patient ***REMOVED*** (except for the sequence from 2015 `viremia blip') to the birth-death serially sampled model in BEAST v1.8.3 \cite{beast} using $10^9$ runs including $10^8$ runs of burn-in.
This gave us birthrate: $\lambda = 8.87 \times 10^{-7}$ day${}^{-1}$, death rate: $\delta = 7.75 \times 10^{-3}$ day${}^{-1}$, and sampling proportion: $s = 0.115$.
We applied a molecular clock to the generated phylogenies using the R package NELSI v \cite{nelsi} with mean rate $4.03 \times 10^{5}$ substitutions per base per day and standard deviation $1.84  \times 10^{-6}$ substitutions per base per day.
These values were obtained from estimated paramater value and standard deviation of $\mu$ in the linear model (LM) given in \cref{eq:glm} using R \cite{rscript}.
Finally, we created simulated HIV RNA sequences from these phylogenies with INDELible v1.03 \cite{indelible}, to create one set of 50 simulated HIV RNA sequences per phylogeny.
\subsection*{Creation of stress data sets}
((Stress testing.))
\subsection*{Blind-dating}
We reconstructed the within-host phylogenies of each subject in each data set as maximum likelihood phylogenies reconstructed using RAxML v8.2.4 \cite{raxml} with the GTR model.
We evaluated two different methods for rooting these trees. 
The first method was outgroup rooting (OGR), in which one or more sequences from taxa outside but sufficiently related to the group of interest are added to the phylogenetic reconstruction \cite{Huelsenbeck02}. and the phylogenetic reconstruction \cite{Huelsenbeck02}.
The point at which the branch leading to the outgroup intersects the rest of the tree provides an estimate of the root for the latter.
We selected the HIV-1 B ancestral sequence reconstruction curated by the LANL HIV sequence database \cite{LosAlamos} as the outgroup sequence.
The second method was root-to-tip regression (RTT) \cite{Korber00}, which performs an exhaustive search by re-rooting the tree at every branch to find the root that (in our experiments) maximizes the correlation between the sample collection dates and the evolutionary distance of tips from the candidate root.
Our simulated data were only rooted using RTT; for all real data, both outgroup rooting and RTT were evaluated for rooting the phylogeny.
For each subject in each data set, we computed the LM given in \cref{eq:glm} using the R function, glm \cite{rscript}.
\subsection*{Statisical Analysis}
All statistical calculations were performed using R \cite{rscript}.
All plots were generated using the ggplot2 and ggtree packages in R \cite{ggplot,ggtree} --- time-scaled versions of the trees were generated using the esimate.dates function of the ape package in R \cite{ape,nodedating}.
}

\showmatmethods{} % Display the Materials and Methods section

%\acknow{Ack}

%\showacknow{} % Display the acknowledgments section

% \pnasbreak splits and balances the columns before the references.
% If you see unexpected formatting errors, try commenting out this line
% as it can run into problems with floats and footnotes on the final page.
\pnasbreak{}

% Bibliography
\bibliography{blind-dating}
\end{document}


%% Do not use widetext if paper is in single column.
\begin{widetext}
\begin{align*}
(x+y)^3&=(x+y)(x+y)^2\\
       &=(x+y)(x^2+2xy+y^2) \numberthis \label{eqn:example} \\
       &=x^3+3x^2y+3xy^3+x^3. 
\end{align*}
\end{widetext}

\begin{table}%[tbhp]
\centering
\caption{Comparison of the fitted potential energy surfaces and ab initio benchmark electronic energy calculations}
\begin{tabular}{lrrr}
Species & CBS & CV & G3 \\
\midrule
1. Acetaldehyde & 0.0 & 0.0 & 0.0 \\
2. Vinyl alcohol & 9.1 & 9.6 & 13.5 \\
3. Hydroxyethylidene & 50.8 & 51.2 & 54.0\\
\bottomrule
\end{tabular}

\addtabletext{nomenclature for the TSs refers to the numbered species in the table.}
\end{table}