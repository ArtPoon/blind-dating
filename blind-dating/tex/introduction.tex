\section * {Introduction} \label{sec:intro}
A major obstacle along the path of developing HIV cure strategies is the existence of latent viral reservoirs within infected patients \citep{Pace11}. 
Reservoirs cells are infected cells that contain integrated viral DNA and have entered a dormant state in which they not only have a lowered rate of virion production, but may also persist within an individual for several years or more among the patient's active viral population.
Although a patient's viral load can be reduced below detectable levels when that patient is subjected to highly active antiretroviral therapy (HAART), latent reservoirs will eventually reseed an infection if a patient halts treatment \citep{Joos08, Pomerantz03, Richman09}. 


Due to selective pressure from the host's immune system and the high mutation rate and short generation time of the virus, HIV-1 evolves very quickly throughout the course of infection \citep{Alizon13, Shankarappa99, Rambaut04}. 
Between the active and latent populations, latent cells are then expected to be more genetically different to active lineages than active lineages are among each other.
Consequently, it is possible to reconstruct a phylogeny that models the pattern of common ancestry from the genetic differences among virus lineages that have been sampled from a given infection. 
With an appropriate method of relating evolutionary distance to time (a {\em clock model}), the phylogeny can be re-scaled chronologically, then by fixing the tips to their respective sampling dates, clock ``rates'' can be estimated \citep{Rodrigo99}.
For sequences originating from the active population of virus, it is not unreasonable to assume that the rate of molecular evolution is relatively constant -- in other words, it adheres to a `molecular clock' model \citep{Leitner99, Kuhner95, Korber00}. 
Once the virus has integrated itself into a cell, however, the rate at which it accumulates mutations becomes negligible compared to the rate at which the active population in plasma evolves. 
Thus, integration transiently halts the evolution of that particular virus lineage. 
Viral latency can be manifested by a discordance between a sample's actual collection time and its apparent age based on its sequence divergence from other lineages (Figure \ref{fig:latenttree}). 
Therefore, it may be possible to use a molecular clock model to reconstruct the dates that a given virus lineage became latent.

Not much is known about when latent reservoirs are established over the course of an infection. 
The chronology of their establishment is of interest as the distribution of ages of the stored virus lineages may provide information concerning the types of adaptations that the virus may have accumulated. 
Specifically, it may have an influence on how those infected cells react to immune-mediated and/or therapeutic treatments. 
Previous studies have looked at the population dynamics of different classes of the virus within host, modeled those classes with systems of ordinary differential equations, then fit those models to empirical data \citep{Althaus14}. 
There has also been work that attempts to characterize latent sequences using a relaxed clock approach \citep{Immonen14}. 
However, there has not been much work towards assigning dates to HIV sequences that are suspected to be latent, or qualitatively assessing the distribution of collection-age differences for suspected latent samples.

We propose a simple framework to estimate dates of latent reservoirs that utilizes the assumption of a strict molecular clock -- that the expected number of substitutions per site is linearly related to time \citep{Ho14} --  on the evolution of HIV within host. 
Our method extracts timing information from the phylogenetic relationship between the virus sampled at various time-points along the infection. 
We first construct a phylogeny containing both cellular HIV DNA and plasma HIV RNA derived sequences from the same patient, then parameterize a strict clock model on that tree based solely on the collection dates of the HIV RNA sequences. 
Since the root of this phylogeny is typically unknown, we employ either root-to-tip regression \citep{Korber00} or outgroup rooting to estimate the location of the root in the tree. 
We assume that sequences derived from cellular HIV DNA such as that extracted from peripheral blood mononuclear cells (PBMCs) are potentially derived from latent reservoirs, and have therefore have effectively stopped evolving.
Although sequences derived from plasma HIV RNA may potentially be descended from reactivated latent virus lineages, we assume that this affects only a minority of sequences derived from blood plasma on the time scales represented by the longitudinal data sets in our study (see below).

To assess the sensitivity of our method, we first simulated sequence data with no viral latency and evaluated the accuracy of predicting sample collection times from a strict molecular clock model that was fit to a training subset of the sequences. 
Next, we evaluated the same model on sequence data that was simulated under a model that included viral latency in the form of a variable molecular clock.
To mimic the types of results we expected from our first test, we used a collection of published longitudinal patient-derived sequence data sets comprising exclusively plasma-derived HIV sequences \citep{McCloskey14} to evaluate the accuracy of date reconstruction for sequences with censored dates. 
Finally, we tested our methodology on another collection of longitudinal patient-derived data sets comprising both PBMC and plasma sequences. 
If the molecular clock assumption holds for all sequences, and latency merely implies a ``pause'' in the evolution of a sequence, then the clock should be a reliable source of information for dating latent sequences.

\begin{figure} \label{fig:latenttree}
	\centering
	\scalebox{5}{%LaTeX with PSTricks extensions
%%Creator: inkscape 0.91
%%Please note this file requires PSTricks extensions
\psset{xunit=.5pt,yunit=.5pt,runit=.5pt}
\begin{pspicture}(85,53)
{
\newrgbcolor{curcolor}{0 0 0}
\pscustom[linewidth=0.38699999,linecolor=curcolor]
{
\newpath
\moveto(77.921134,4.13898)
\lineto(5.1912118,4.13898)
\lineto(5.1912118,20.207747)
}
}
{
\newrgbcolor{curcolor}{0 0 0}
\pscustom[linewidth=0.38699999,linecolor=curcolor]
{
\newpath
\moveto(65.799478,25.564002)
\lineto(29.434525,25.564002)
\lineto(29.434525,36.276514)
}
}
{
\newrgbcolor{curcolor}{0 0 0}
\pscustom[linewidth=0.38699999,linecolor=curcolor]
{
\newpath
\moveto(41.556173,46.989025)
\lineto(29.434525,46.989025)
\lineto(29.434525,36.276514)
}
}
{
\newrgbcolor{curcolor}{0 0 0}
\pscustom[linewidth=0.38699999,linecolor=curcolor]
{
\newpath
\moveto(29.434525,36.276514)
\lineto(5.1912118,36.276514)
\lineto(5.1912118,20.207747)
}
}
{
\newrgbcolor{curcolor}{0 0 0}
\pscustom[linewidth=0.38699999,linecolor=curcolor]
{
\newpath
\moveto(4.4639148,20.207747)
\lineto(5.1912118,20.207747)
}
}
{
\newrgbcolor{curcolor}{0 0 0}
\pscustom[linestyle=none,fillstyle=solid,fillcolor=curcolor]
{
\newpath
\moveto(81.87934379,5.47198716)
\lineto(81.87934379,5.02049697)
\curveto(81.73520608,5.15474288)(81.58117656,5.25507403)(81.41725524,5.32149043)
\curveto(81.25474704,5.38790683)(81.08164047,5.42111503)(80.89793554,5.42111503)
\curveto(80.53617814,5.42111503)(80.25920764,5.31018551)(80.06702402,5.08832648)
\curveto(79.8748404,4.86788057)(79.77874859,4.54851662)(79.77874859,4.13023462)
\curveto(79.77874859,3.71336575)(79.8748404,3.3940018)(80.06702402,3.17214277)
\curveto(80.25920764,2.95169685)(80.53617814,2.84147389)(80.89793554,2.84147389)
\curveto(81.08164047,2.84147389)(81.25474704,2.87468209)(81.41725524,2.94109849)
\curveto(81.58117656,3.00751489)(81.73520608,3.10784604)(81.87934379,3.24209195)
\lineto(81.87934379,2.79484111)
\curveto(81.72955362,2.69309684)(81.5705782,2.61678864)(81.40241754,2.5659165)
\curveto(81.23566999,2.51504437)(81.05903063,2.4896083)(80.87249947,2.4896083)
\curveto(80.39345355,2.4896083)(80.01615188,2.63586569)(79.74059449,2.92838046)
\curveto(79.4650371,3.22230834)(79.3272584,3.6229264)(79.3272584,4.13023462)
\curveto(79.3272584,4.63895596)(79.4650371,5.03957402)(79.74059449,5.33208879)
\curveto(80.01615188,5.62601668)(80.39345355,5.77298062)(80.87249947,5.77298062)
\curveto(81.06185686,5.77298062)(81.23990933,5.74754455)(81.40665688,5.69667242)
\curveto(81.57481755,5.6472134)(81.73237985,5.57231831)(81.87934379,5.47198716)
\closepath
}
}
{
\newrgbcolor{curcolor}{0 0 0}
\pscustom[linestyle=none,fillstyle=solid,fillcolor=curcolor]
{
\newpath
\moveto(67.94040944,25.49054575)
\lineto(67.94040944,24.41103725)
\lineto(68.57982581,24.41103725)
\curveto(68.79428027,24.41103725)(68.95281869,24.45511225)(69.05544107,24.54326224)
\curveto(69.15937912,24.63272791)(69.21134815,24.76889991)(69.21134815,24.95177825)
\curveto(69.21134815,25.13597227)(69.15937912,25.27148644)(69.05544107,25.35832076)
\curveto(68.95281869,25.44647075)(68.79428027,25.49054575)(68.57982581,25.49054575)
\lineto(67.94040944,25.49054575)
\closepath
\moveto(67.94040944,26.70227923)
\lineto(67.94040944,25.81420095)
\lineto(68.53048813,25.81420095)
\curveto(68.72520752,25.81420095)(68.86993138,25.85038191)(68.96465973,25.92274385)
\curveto(69.06070376,25.99642145)(69.10872577,26.10825353)(69.10872577,26.25824009)
\curveto(69.10872577,26.40691097)(69.06070376,26.51808522)(68.96465973,26.59176282)
\curveto(68.86993138,26.66544043)(68.72520752,26.70227923)(68.53048813,26.70227923)
\lineto(67.94040944,26.70227923)
\closepath
\moveto(67.54176097,27.02988145)
\lineto(68.56009074,27.02988145)
\curveto(68.86401086,27.02988145)(69.0982004,26.96672921)(69.26265934,26.84042474)
\curveto(69.42711828,26.71412028)(69.50934775,26.53453111)(69.50934775,26.30165725)
\curveto(69.50934775,26.12141025)(69.46724626,25.97800205)(69.38304328,25.87143266)
\curveto(69.29884031,25.76486326)(69.17516718,25.69842185)(69.01202391,25.67210842)
\curveto(69.20805897,25.63000693)(69.36001903,25.54185694)(69.4679041,25.40765844)
\curveto(69.57710484,25.27477562)(69.6317052,25.10834317)(69.6317052,24.90836109)
\curveto(69.6317052,24.64522679)(69.54223954,24.44195553)(69.36330821,24.29854734)
\curveto(69.18437688,24.15513914)(68.92979444,24.08343504)(68.59956088,24.08343504)
\lineto(67.54176097,24.08343504)
\lineto(67.54176097,27.02988145)
\closepath
}
}
{
\newrgbcolor{curcolor}{0 0 0}
\pscustom[linestyle=none,fillstyle=solid,fillcolor=curcolor]
{
\newpath
\moveto(68.8325765,47.98448198)
\lineto(68.33071752,46.62360158)
\lineto(69.33626709,46.62360158)
\lineto(68.8325765,47.98448198)
\closepath
\moveto(68.62377386,48.34897081)
\lineto(69.04321075,48.34897081)
\lineto(70.08539237,45.6143888)
\lineto(69.70075592,45.6143888)
\lineto(69.45165803,46.31589242)
\lineto(68.21898978,46.31589242)
\lineto(67.96989189,45.6143888)
\lineto(67.57976063,45.6143888)
\lineto(68.62377386,48.34897081)
\closepath
}
}
{
\newrgbcolor{curcolor}{0 0 0}
\pscustom[linewidth=0.39463133,linecolor=curcolor,linestyle=dashed,dash=0.7892628 0.7892628]
{
\newpath
\moveto(41.217255,46.97727)
\lineto(65.885506,46.9308)
}
}
\rput(42,52){\psscalebox{0.3}{$t$}}
\rput(66,52){\psscalebox{0.3}{$t^\prime$}}
\end{pspicture}
}
	\caption[Example of latent behavior]{ Above is a time calibrated phylogeny with three taxa, the dotted line in is an example of latency. 
	Sequence A was archived at time $t$, and was collected at the same time as sequence B, at time $t^\prime$ (a drift of $t^\prime - t$ is expected).
	If the molecular clock assumption holds true, the MRCA is known, and B and C are reliable time-points, then the date at $t$ can be inferred from from the expected amount of evolution for B from the root.}
\end{figure}
